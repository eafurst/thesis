\chapter{Conclusion}\label{gen:sec:conclusion}
\markboth{Conclusion}{Conclusion}

This thesis presents solutions to the challenge of writing peformant code for graph applications on a manycore architecture. 
The optimizations presented in this work aim to better leverage the memory hierarchy and architectural features of a manycore in order to achieve high performance on graph applications. 
In this thesis, I present a code generation backend to the \graphit DSL that allows for flexibility in application development and optimization. 
I show how existing optimizations can be implemented on a manycore in order to improve work efficiency and decrease workload imbalance across cores. 
I present several manycore specific optimizations that target the memory hierarchy in order to more efficiently use memory bandwidth.
%I further propose two areas of future work to increase the efficiency and performance of the code produced by my \graphit code generation backend. 

As the size of sparse graph data continues to grow, the importance of scalable, performant graph processing systems will only continue to increase. 
My thesis demonstrates ways that we can leverage the parallelism offered by emerging manycore architectures in order to maintain scalable performance on graph applications while also maintaining the flexibility and ease of programming offered through the use of the \graphit DSL. 
As graph processing demands grow and users turn more and more to emerging architectures, I expect that future work will continue to build on these efforts to reduce programming complexity through code generation and increase performance through optimizations that target architectural features. 

\section{Future Work}
%\todo{areas of research that could be interesting in this space}
There are several areas of exciting future work to be done in the space of graph processing on the \hb~manycore architecture. 
Much of the future work in this space focuses on better understanding application performance and on improving performance through continued memory system and data layout optimizations.
I believe that the code generation framework that this thesis presents will provide the flexibility and extensibility necessary to explore these areas of future work.

 \paragraph{Multi-Source Traversal}\mbox{}\\
Breadth-first search (BFS) is a common building block of many graph algorithms, and there has been substantial effort in optimizing BFS on parallel architectures. 
While quite a few algorithms such as SSSP execute BFS from a single source vertex, there also exist many algorithms which execute many different BFS traversals from different source vertices including all pairs shortest paths and betweenness centrality.
There has been limited work optimizing this multi-souce BFS (MS-BFS) for parallel architectures~\cite{then2014more, liu2016ibfs}.

The approach outlined by Then et al.~\cite{then2014more} leverages the properties of small-world networks that are present in many real-world graphs.
They aim to optimize concurrent processing of a large number of BFS traversals in a single core.
The authors observe that in graphs that exhibit small-world properties, the majority of vertices are discovered in just a few iterations.
Further, they note that multiple BFS traversals over the same graph have a high likelihood of having overlapping sets of discovered vertices within the same iteration.
Based on these observations, they create data structures that allow for combined accesses to the same vertices across multiple BFS instances.
They use sets and set operations to concurrently execute many BFS traversals at the same time. 
In practice, these are implemented as bit operations over fixed width bit fields where the width of the bit field is the maximum number of concurrent BFS traversals that are supported.
This approach reduces cache-contention and redundant work.

We believe that the manycore will be well suited to this style of MS-BFS processing, and that the use of bitfields will work well with the small caches and scratchpad memories present on the manycore.
In addition, \graphit has support for multi-source traversal through its \lstinline{parallel_for} operator. 
However, this is a feature that we do not currently support in our \hb~code generation backend. 
An optimal implementation of this operator could open up our code generation framework to a larger class of graph applications.

\paragraph{Parameter Tuning and Custom Data Structures}\mbox{}\\
In graph processing, proposed optimizations are often tuned for one specific architecture.
This means that when an optimization is ported to a new architecture, a performance study must be conducted to tune the optimization parameters to the specific features of the new architecture.
We began a study of parameter tuning in this thesis with the exploration of hybrid traversal methods. 
From this we found that for optimal performance \hb required smaller $\alpha$ and $\beta$ values for the hybrid traversal heuristics introduced by Beamer et al.

Another area to evaluate would be tuning the $\Delta$ parameter in the delta-stepping SSSP algorithm.
As discussed in Chapter~\ref{gen:sec:background}, delta-stepping is a variant that was proposed to increase parallelism without dramatically increasing algorithmic work by introducing a bucketed priority queue~\cite{meyer2003delta}. 
The bucketed priority queue increases parallelism by relaxing the strict processing order imposed in Dijkstra's algorithm.
Vertices are sorted into buckets within the priority queue of size $\Delta$.
Setting $\Delta = 1$ effectively turns delta-stepping into Dijkstra's and setting $\Delta = \infty$ turns it into Bellman-Ford. 

It has been noted that the $\Delta$ parameter is sensitive to both input graph and architecture~\cite{beamer2016thesis}.
Beamer et al. found that the range of optimal $\Delta$ values for a given input graph is actually quite large~\cite{beamer2016thesis}. 
However, there has been less study on how sensitive $\Delta$ values are to different architectures. 
A study of $\Delta$ values on road network and social network graphs to determine optimal values for the \hb~manycore could provide interesting insights into the class of ordered graph algorithms.

In addition, there is room to improve the implementation of the bucketed priority queue for a manycore architecture.
Currently the priority queue is implemented on the host and all bucket updates must happen on the host in between each iteration.
At the beginning of each iteration, the priority queue constructs a frontier to be dispatched to the manycore for processing.
We believe that the manycore could be well suited to a device side implementation of the priority queue, and that moving the logic of the priority queue to the device could greatly increase performance.