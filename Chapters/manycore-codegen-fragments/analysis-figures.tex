\newcommand{\tabHBParams}{
%\newcolumntype{D}{>{\raggedleft\arraybackslash}m{0.65in}}
%\newcolumntype{E}{>{\raggedright\arraybackslash}m{2.75in}}
\newcolumntype{D}{>{\raggedleft\arraybackslash}m{0.6175in}}
\newcolumntype{E}{>{\raggedright\arraybackslash}m{2.7175in}}
\begin{table}[ht]
  \centering
  %\vspace{-0.1in}
  \resizebox{0.7\linewidth}{!}{
  \begin{footnotesize}
    \begin{tabular}{DE}
      \toprule
      \textbf{Cores} & 128 cores in a 16x8 grid \\
      & RISC-V 32-bit IMAF ISA @ 1 GHz\\
      & 4KB Instruction Cache \\
      & 4KB Data Scratchpad \\ %%[.5ex]
      \midrule
      \textbf{Cache} & 128KB Total Capacity \\
      & 32 Independent Cache Banks \\
      & 8-way Set Associative\\ [.2ex]
      \midrule
      \textbf{NoC} & Bidirectional 2D Mesh NoC (32-bit data, 64-bit address) \\ [.2ex]
      \midrule
      \textbf{Memory} & 2 HBM2 memory channels \\
      & 32 GB/s per channel \\
      & 512 MB per channel \\
      \bottomrule
    \end{tabular}
  \end{footnotesize}
  }
  \caption{Configuration of the \hbmc used in this evaluation.}
  \label{tab:hbParams}
\end{table}
}

\newcommand{\evalGraphsTab}{
\begin{table}[ht]
  %\arraystretch{1.3}
  \centering
  \begin{footnotesize}
  \begin{tabular}{lccc}
  %\hline
  \toprule
    \textbf{Name} & \textbf{Scale} & \textbf{\# Vertices} & \textbf{\# Edges} \\ \midrule %\hline
    kron18 & 18 & 262,144 & 4,194,304 \\ %\hline
    kron19 & 19 & 524,288 & 8,388,608 \\
    kron20 & 20 & 1,048,576 & 16,777,216 \\ %\hline
    kron22 & 22 & 4,194,304 & 67,108,864 \\ %\hline
    Pokec (pk) & 20.5 & 1,632,803 & 30,622,564 \\ %\hline
    LiveJournal (lj) & 22 & 3,997,962 & 34,681,189 \\ %\hline
    Hollywood (hw)& 20.1 & 1,139,905 & 112,751,422 \\
    RoadCA (rc)& 20.9 & 1,971,281 & 5,533,214 \\
    RoadCentral (rn)& 23.8 & 14,081,816 & 33,866,826 \\
    RoadUSA (ru)& 24.5 & 23,947,347 & 57,708,624\\
    \bottomrule
  \end{tabular}
  \end{footnotesize}
  \caption{The vertex and edge information for each of the graphs used in our evaluation. We use synthetic \kron graphs used in the Graph500 benchmark and six real world graphs.}
  \label{sec:eval:tab:graphs}
\end{table}
}

\newcommand{\hbBlockingEvalTab}{
\begin{table}[ht]
\centering
\begin{footnotesize}
\begin{tabular}{lccc}
\toprule
\textbf{Graph} & \textbf{DRAM Stalls} & \textbf{Bandwidth} &  \textbf{Speedup} \\ \midrule
LiveJournal & 0.78 & 3.03 & 1.19    \\
Hollywood &  0.79 & 2.17 & 1.53   \\
Pokec & 0.83 & 3.02 & 1.49     \\
\bottomrule
\end{tabular}
\end{footnotesize}
  \caption{Impact of the blocked access optimization on SSSP. Reduction in DRAM stalls, improvement in memory bandwidth utilization, and overall speedup. } %over the unoptimized version of \sssp. }%Results are shown for the three graphs for which the blocked access method was the optimal schedule.}
\label{tab:hb_blocking}
\end{table}
}

\newcommand{\edgeAwareHist}{
\begin{sidewaysfigure}
    \centering
    \subfloat[BFS 18]{\includegraphics[width=0.3\textwidth]{graphit-figures/bfs-pull-18-edge-aware.png}}
    \subfloat[BFS 19]{\includegraphics[width=0.3\textwidth]{graphit-figures/bfs-pull-19-edge-aware.png}}
    \subfloat[BFS 20]{\includegraphics[width=0.3\textwidth]{graphit-figures/bfs-pull-20-edge-aware.png}} \par\vspace{-4.5mm}%
    \subfloat[SSSP 18]{\includegraphics[width=0.3\textwidth]{graphit-figures/sssp-pull-18-edge-aware.png}}
    \subfloat[SSSP 19]{\includegraphics[width=0.3\textwidth]{graphit-figures/sssp-pull-19-edge-aware.png}}
    \subfloat[SSSP 20]{\includegraphics[width=0.3\textwidth]{graphit-figures/sssp-pull-20-edge-aware.png}} \par\vspace{-4.5mm}%
    \subfloat[PR 18]{\includegraphics[width=0.3\textwidth]{graphit-figures/pr-pull-18-edge-aware.png}}
    \subfloat[PR 19]{\includegraphics[width=0.3\textwidth]{graphit-figures/pr-pull-19-edge-aware.png}}
    \subfloat[PR 20]{\includegraphics[width=0.3\textwidth]{graphit-figures/pr-pull-20-edge-aware.png}} 
    \vspace{-2.5mm}
    \caption{Probability distribution of cycle counts for each core for vertex-based and edge-aware partitioning.}
    \label{fig:load_balance:cycledist}
\end{sidewaysfigure}
}

\newcommand{\alphatune}{
  \begin{figure}[!ht]
    \centering
    \includegraphics[scale=0.65]{graphit-figures/alpha.pdf}
    \caption{Performance results for hybrid traversal on BFS for each graph relative to its best performance for a range of $\alpha$ values. We find $\alpha = 5$ to be optimal on \hb.}% \todo{finish caption}}
    \label{fig:alpha_tuning}
  \end{figure}
}

\newcommand{\betatune}{
  \begin{figure}[!ht]
    \centering
    \includegraphics[scale=0.65]{graphit-figures/beta.pdf}
    \caption{Performance results for hybrid traversal on BFS for each graph relative to its best performance for a range of $\beta$ values. We find $\beta = 4$ to be optimal for these graphs on \hb}% \todo{finish caption}}
    \label{fig:beta_tuning}
  \end{figure}
}

\newcommand{\hybridresults}{
  \begin{figure}[!ht]
    \centering
    \includegraphics[scale=0.65]{graphit-figures/hybrid-dir.pdf}
    \caption{Traversal direction speedups for BFS relative to the \push implementation. Hybrid-beamer is the hybrid method proposed by Beamer et al. and hybrid-ligra is the method introduced in Ligra.}
    \label{fig:hybrid_dir}
  \end{figure}
}