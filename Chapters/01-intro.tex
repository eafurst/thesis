\chapter{Introduction}\label{gen:sec:intro}
\markboth{Introduction}{Introduction}
\newcommand{\introOverviewFigure}{
\begin{figure}[t!]
    \centering
    \includegraphics[scale=0.5]{graphit-figures/overview.pdf}
    \caption{This figure shows the flow from Graph Algorithm (e.g. PageRank) to GraphIt code to execution on the manycore architecture. 
    Code is generated for both the host and the device. The graph data structure is loaded into device memory by the host program, and the device program is executed in parallel on the manycore architecture.}
    \label{pap:generals:sec:intro:fig:overview}
\end{figure}
}

%\todo{at least one more paragraph here describing graph processing and its importance}

Sparse graph data structures that capture relationships between elements are ubiquitous.
In recent years, the size of graph datasets have exploded~\cite{sahu2017ubiquity} with data coming from fields like networking~\cite{lehmberg2014structure}, social networks~\cite{sharma2016graphjet, eksombatchai2018pixie}, public health~\cite{keeling2005networks, Ulyantsev2016Metafast}, chemistry \cite{tsubaki2019compound}, and finance~\cite{boginski2005finance}. 
As graph sizes have grown, so too have the processing requirements. Recommender systems must respond in sub-second latencies \cite{sharma2016graphjet, eksombatchai2018pixie}, and financial algorithms trade on the order of microseconds \cite{menkveld2018hft}.
These challenges have increased the demand for high-performance graph processing systems.

However, implementing high-performance graph processing systems is not as straightforward as simply increasing the number of compute cores or adding more memory bandwidth. 
Further, optimizations used in serial applications or implementations with limited parallelism often do not continue to provide performance when parallelism is increased~\cite{beamer2015locality}.
This has resulted in a wide design space for algorithmic and architectural optimizations to increase the performance of these parallel graph processing systems at scale.


\section{Challenges for Parallel Graph Processing}

Graph processing is notoriously difficult to optimize. 
Performance depends on optimizing locality within sparse data structures, random-access performance, minimizing high-cost communication, and load balancing between parallel threads~\cite{lumsdaine2007challenges, beamer2015locality}. 
Worse, the structure of graphs varies widely, both between graphs and between iterations within graph algorithms~\citep{lumsdaine2007challenges, beamer-bfs-direction}. Heuristic optimizations do not benefit all inputs datasets. 
Therefore, graph frameworks must be descriptive enough, and processing hardware must be flexible enough to support them.

These challenges have led to the development of many parallel graph processing frameworks: GraphIt~\cite{zhang2018graphit}, GraphLab~\cite{low2010graphlab, low2012distributed}, Grappa~\cite{nelson2015grappa}, GraphChi~\cite{aapo2012graphchi}, and Pregel~\cite{malewicz2010pregel}. 
These frameworks take a user application description and emit parallel code for general-purpose hardware.
Traditionally graph frameworks have targeted CPU and GPU architectures, but newer frameworks also target cloud-resident FPGAs \cite{engelhardt2016gravf, dai2016fpgp}.
These frameworks allow users to focus on exploring optimizations by abstracting the hardware and parallel infrastructure.

Graph processing frameworks are limited by the flexibility and parallelism provided by the hardware they currently target.
Server-class CPUs are widespread and support flexible execution models, but they have limited memory and compute parallelism and poor random-access bandwidth~\citep{beamer2015locality}.
GPUs are also widespread and expose memory parallelism through banking and multiple memory channels, but are limited by vector-like execution models \cite{xu2014graph, shi2018graph}, and have poor random-access bandwidth~\citep{aamodt2018general}.
FPGAs are inherently more flexible than either architecture, but are difficult to optimize when a single recompilation can take hours. 
  
% Solution       
Manycore architectures are composed of hundreds to thousands of efficient general-purpose cores to form a flexible parallel compute fabric.
Past manycore architectures~\cite{ramey2011tilera, agathos2015parallela, gwennap2011adapteva} have been limited by available memory bandwidth and parallelism~\citep{loi2010efficient}.  We employ a manycore design connected to High Bandwidth Memory (HBM) \cite{jedec2020hbm, jouppi2017datacenter} to overcome this issue.
In my thesis, I will show how this manycore architecture can be leveraged to provide high performance on a variety of graph processing applications.


\section{Thesis Overview}
My thesis proposes a code generator for graph programs targeting a representative manycore architecture. 
Within this code generator, I propose and implement manycore specific optimizations to improve the graph processing performance. 
\todo{finish rewriting this section}
\section{Organization}
\todo{give a roadmap of each chapter and its contributions/contents}
%While graph processing frameworks and optimizations on parallel systems is an active area of research (\Cref{gen:sec:relatedwork}), my proposal to target a novel manycore and my subsequently proposed optimizations for this architecture expand the space and provide a new class of techniques for graph applications on this emerging architecture. 

%In this thesis, I focus on two different aspects of this problem: ease of programming and manycore specific optimizations. I use information gained through experimentation with the manycore architecture to optimize graph applications (\Cref{sec:method}) and build these optimizations into a code generation engine (\Cref{sec:method:sub:baseline}).

%\begin{itemize}
%\item \textbf{Ease of Programming} Reasoning about the implementation of a graph application and the details of the underlying architecture can be difficult especially as techniques for obtaining performance on architectures become increasingly complex. I address this by proposing a code generator that allows the user to reason about the graph application and optimizations at a high level without requiring knowledge of the underlying manycore architecture (\Cref{sec:method:sub:baseline}).
%\item \textbf{Manycore Specific Optimizations} I use observations about the manycore architecture to implement existing graph processing optimizations and propose several manycore specific optimizations that build off of the observation that most graph applications are memory bound (\Cref{sec:method}). I propose further study of performance on the manycore in order to optimize a variety of techniques and applications (\cref{gen:sec:proposed}).
%\end{itemize}

%I first contextualize my work within the space of graph processing frameworks and optimizations in Section~\ref{gen:sec:relatedwork}.
%Then, I provide background information on the \graphit DSL and the representative manycore architecture that I target in Section~\ref{gen:sec:background}.
%In Section~\ref{gen:sec:graphitbackend}, I present my work implementing a code generation backend for the \graphit DSL. 
%I further propose manycore specific optimizations in Section~\ref{gen:sec:graphitbackend} and discuss the implementation of several other graph processing optimizations. 
%Finally, Section~\ref{gen:sec:proposed} presents two areas of work that I plan to pursue in the completion of my thesis: a study and analysis of tunable parameters to find optimal values for various graph optimizations and a manycore implementation of multi-source BFS.


% In this paper, we outline a general execution model for this class of manycore systems.
% We then describe a prototypical computational model for graph processing applications that provides computation and memory parallelism.
% %Manycore architectures introduce a level of complexity that can make it difficult to reason about how best to obtain performance in graph application code.
% It can be challenging to write code for manycore architectures that exploits the
% right types of parallelism to maximize performance for an application.

